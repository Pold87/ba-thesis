% Created 2014-03-17 Mon 02:22
\documentclass[koma,DIV=1,BCOR=1mm]{article}
\usepackage[utf8]{inputenc}
\usepackage[T1]{fontenc}
\usepackage{fixltx2e}
\usepackage{graphicx}
\usepackage{longtable}
\usepackage{float}
\usepackage{wrapfig}
\usepackage{rotating}
\usepackage[normalem]{ulem}
\usepackage{amsmath}
\usepackage{textcomp}
\usepackage{marvosym}
\usepackage{wasysym}
\usepackage{amssymb}
\usepackage{hyperref}
\tolerance=1000
\author{Volker Strobel}
\date{\today}
\title{Skript für Tutorial}
\hypersetup{
  pdfkeywords={},
  pdfsubject={},
  pdfcreator={Emacs 24.3.1 (Org mode 8.2.5h)}}
\begin{document}

\maketitle
\section{Einführung}
\label{sec-1}
Dieses Videotutorial soll in das Thema Planung und vorallem in die
Planungssprache PDDL einführen. Planung ist eine der wichtigsten
Komponenten des Problemlösens. Ein Problem liegt dann vor, wenn der
Soll-Zustand nicht dem Ist-Zustand entspricht. Bei der Planung wird
eine Abfolge von Handlungsschritten, sogenannten Aktionen entworfen,
die vom Ist-Zustand zum Soll-Zustand führt. D.h. es wird ein Plan wird
erstellt, der das Problem löst. Ziel der künstlichen Intelligenz
Forschung ist es natürlich, dass die Planung von der künstlichen
Intelligenz ausgeführt wird. Damit dies erfolgen kann, ist eine
detaillierte Beschreibung der Umwelt und des Problems notwendig.
Ausserdem bedarf es eines Algorithmus, der aus diesen Beschreibungen
die nötigen Schritte herleiten kann, die zur Lösung des Problems
führen. Die drei Bestandteile der automatisierten Planung sind also:

\begin{table}[htb]
\caption{Planung in der KI}
\centering
\footnotesize
\begin{tabular}{l|lll}
\textbf{Komponente} & \textbf{Domäne} & \textbf{Problem} & \textbf{Planer}\\
\hline
\textbf{Definition} & Beschreibung der Welt & Beschreibung des Problems & Software, die aus der\\
 & - Eigenschaften und Beziehungen von Objektypen & - Instanziierung der Welt & Beschreibung der Welt und\\
 & - Einschränkungen &  & \\
 & - mögliche Aktionen und ihre Konsequenzen & - Soll-Zustand & entwickelt\\
 &  & - Ist-Zustand & des Problems einen Plan\\
 &  &  & \\
 &  &  & \\
 &  &  & \\
\end{tabular}
\end{table}


Die Domäne und das Problem müssen zunächst erstellt werden, bevor sie
dem Planer übergeben werden können. Dieses Modellieren wird heutzutage
noch hauptsächlich von menschlichen Experten vorgenommen. Daher bedarf
eines Formulierungsstandards, der durch Planungssprachen gewährleistet
wird. Analog zu Programmiersprachen haben diese eine Syntax, also
konkrete Regeln und Vorschriften zum Erstellen von Domänen und
Problemen. Eine dieser Planungssprachen ist die Planning Domain
Definition Language oder PDDL.

\begin{table}[htb]
\caption{Planung in der KI}
\centering
\footnotesize
\begin{tabular}{l|lll}
\textbf{Komponente} & \textbf{Domäne} & \textbf{Problem} & \textbf{Planer}\\
\hline
\textbf{Definition} & Beschreibung der Welt & Beschreibung des Problems & Software, die aus der\\
 & - Struktur & - Instanziierung der Welt & Beschreibung der Welt und\\
 & - Einschränkungen & - Ist-Zustand & des Problems einen Plan\\
 & - mögliche Aktionen und & - Soll-Zustand & entwickelt\\
 & ihre Konsequenzen &  & \\
\textbf{Umsetzung} & Mit einer Planungssprache & Mit einer Planungssprache & Software\\
 & in einem Editor & in einem Editor & \\
\textbf{Ausführung} & Menschlicher Experte & Menschlicher Experte & KI\\
\end{tabular}
\end{table}


Im Folgenden soll dir nun beigebracht werden wie man mit PDDL Domänen
aufbaut. Dies wird anhand von einem konkreten Beispiel geschehen und
nach jedem Schritt wird dir eine Verständnisfrage gestellt. Du kannst
das Tutorial jederzeit pausieren um Fragen zu stellen, nochmal über
etwas nachzudenken oder zurückspulen wenn du etwas nochmal anschauen
möchtest. Versuche vorallem die Konzepte zu verstehen, damit du später
selbst Domänen erstellen kannst. Es ist nicht von Bedeutung, dass du
die Programmgerüste oder programmspezifischen Begriffe auswendig
lernst, jedoch, dass du sie später verstehst, wenn du sie erneut
siehst.

\section{Beispiel}
\label{sec-2}
\subsection{Die Welt}
\label{sec-2-1}
Willkommen in unserer Welt. Unsere Welt besteht aus zwei Menschen:
Tyler und Marla und einem Roboter, der mit ihnen wohnt aber nichts zur
Sache tut. Tyler arbeitet und Marla studiert Psychologie. Tyler muss
also jeden Tag auf die Arbeit und Marla geht freiwillig jeden Tag zur
Uni. Tyler und Marla wohnen im selben Haus, das viele verschiedene
Räume hat. So gibt es zum Beispiel eine Küche, ein Bad, zwei
Schlafzimmer, ein Wohnzimmer und eine Garage. Außerdem haben sowohl
Tyler als auch Marla ein Auto, aber Marla hat zusätzlich noch ein
Fahrrad. Alle Fahrzeuge stehen in der Garage. Marlas Auto war sehr
günstig und der Motor springt immer an kalten Tagen nicht an. Heute
ist ein kalter Montag Morgen.
\subsection{Das Problem}
\label{sec-2-2}
Ist: Tyler und Marla sind zu Hause.
Soll: Tyler soll auf die Arbeit und Marla soll zur Uni.
\subsection{Aufbau einer Domäne}
\label{sec-2-3}
Wie können wir diese Welt jetzt in PDDL repräsentieren?
PDDL-Dateien beschreiben entweder eine Domäne oder ein Problem, aber immer nur maximal eines.
Jede Domäne in PDDL hat die folgende Form:

\begin{verbatim}
(define (domain name)
	(:requirements :requirement1
 			:requirement2...)
	(:types  subsubtype1 subsubtype2 – subtype1
subtype1 – type1
subtype2 – type2…
subsubtype1 subsubtype2 subtype1 subtype2 type1 type2 …  - object)
	(:predicates  (predicateName1 ?var1 – typeOfVar1)
(predicateName2 ?var2 – typeOfVar2 ?var3 – typeOfVar3) ...)
	(:action actionName1
:parameters (...)
:precondition (...)
:effect (...) )
(:action actionName2
:parameters (...)
:precondition (...)
:effect (...) )
... )
\end{verbatim}
Das wollen wir nun absatzweise anschauen und aufbauen.

\subsubsection{(define (domain \ldots{}))}
\label{sec-2-3-1}

\begin{verbatim}
(define (domain ...))
\end{verbatim}

Wer schon in LISP oder abgeleiteten Programmiersprachen programmiert
hat, dem sollte die Klammerung vertraut vorkommen. Auch das
Schlüsselwort define, das eine PDDL-Datei einleitet, sollte aus LISP
bereits bekannt sein. Als nächstes wird eingeklammert was man
eigentlich in dieser Datei definieren möchte. Dazu muss man sich als
erstes entscheiden ob man eine Domäne oder ein Problem modellieren
will. Wir interessieren uns momentan nur für Domänen, weshalb das
erste Wort immer domain sein wird. Jetzt kann man einen Namen für die
Domäne festlegen. Dieser kann frei gewählt werden. Da die Hauptakteure
unserer Welt Tyler und Marla heißen, nennen wir unsere Domäne
fightclub. Bisher sieht unsere PDDL-Datei also so aus:

\begin{verbatim}
(define (domain fightclub)
\end{verbatim}
\subsubsection{Requirements}
\label{sec-2-3-2}
\begin{verbatim}
 (:requirements  :requirement1
:requirement2...)
\end{verbatim}
Die requirements sind vergleichbar mit dem Preamble in \LaTeX{}. Hier
werden Domäneneigentschaften festgelegt. Unsere Domänen haben nur ein
requirement, nämlich typing. Das heißt, dass es in unseren Welten
Typenhierarchien gibt. Alle weiteren möglichen requirements sollen uns
nicht interessieren. Die Domäne sieht also jetzt so aus:
\begin{verbatim}
(define (domain fightclub)
	(:requirements :typing)
\end{verbatim}
\subsubsection{Types}
\label{sec-2-3-3}
\begin{verbatim}
(:types  subsubtype1 subsubtype2 – subtype1
subtype1 – type1
subtype2 – type2…
type1 type2 …  - object)
\end{verbatim}

Hier wird es jetzt interessant. In diesem Absatz wird die Welt
typisiert, bzw. strukturiert. Alle Entitäten in Domänen sind immer vom
Typ Objekt. Innerhalb dieser Objektoberklasse gibt es dann
verschiedene Typen und Untertypen. Die Hierarchien werden im Absatz
types modelliert. Damit wir nichts vergessen, ist es sinnvoll uns
zunächst den Aufbau unserer Beispielwelt in Form eines Diagramms zu
veranschaulichen.

(UML Diagramm zeichnen)

Nun wollen wir das in PDDL umsetzen:

\begin{verbatim}
(:types man woman – human
	human robot – agent 
	car bike – vehicle
	office university home – building
	kitchen livingroom bedroom garage bathroom – room
	room – home 
	building  – location 
	agent vehicle location – object)
\end{verbatim}

Unsere domain sieht jetzt also so aus:

\begin{verbatim}
(define (domain fightclub)
	(:requirements :typing)
	(:types man woman – human
	human robot – agent 
	car bike – vehicle
	office university home – building
	kitchen livingroom bedroom garage bathroom – room
	room – home 
	building  – location 
	agent vehicle location – object)
\end{verbatim}

Zur Überprüfung ob du das verstanden hast, versuche nun bitte folgende
Hierarchie selbst umzusetzen. Male hierfür zunächst auch ein
UML-Diagramm.

In einer Firma gibt es verschiedene Hierarchieebenen. Ausschließlich
Agenten können bei einer Firma arbeiten. Agenten können entweder
Roboter oder Menschen sein. Während es bei den Menschen
Hierarchiestrukturen gibt, ist dies bei den Robotern nicht der Fall.
Roboter haben keine Chefs und sind auch nicht in Führungspositionen,
sie tun einfach das wofür sie programmiert wurden. Die
Firmenhierarchie ist wie folgt: an oberster Stelle ist der Vorstand.
Neben dem Vorstand gibt es noch den Aufsichtsrat, der ebenfalls nur
aus Menschen besteht. Dem Vorstand ist alles untergeordnet. Auf der
nächsten Ebene lässt sich die Firma nun in die drei Bereiche
Entwicklung, Fertigung und Verkauf einteilen. Diese sind jeweils
wieder in zwei Abteilungen unterteilt, außer der Verkauf, der nur aus
5 einzelnen Arbeitern besteht (einzelne Arbeiter müssen nicht
modelliert werden). Jede Abteilung besteht wiederum aus zwei kleineren
Gruppen.

Aber warum ist das wichtig? Warum reicht es nicht einfach alle Objekte
in der Welt zu benennen? Warum müssen sie auch noch typisiert werden?

Das soll alles mit dem nächsten Absatz klarer werden.
\subsubsection{Predicates}
\label{sec-2-3-4}
\begin{verbatim}
(:predicates  (predicateName1 ?var1 – typeOfVar1)
(predicateName2 ?var2 – typeOfVar2 ?var3 – typeOfVar3) ...)
\end{verbatim}

Um eine Welt zu modellieren, reicht es nicht einfach nur zu sagen, was
alles vorhanden ist, und in welcher Beziehung die Objekte zueinander
stehen. Nein, meistens können Objekte in verschiedenen Zuständen
existieren und diese Zustände können sich ändern. So sind manche
Menschen schizophren und andere nicht, manche Räume haben Außenwände
aus Glas und viele nicht, während einige Männer sich gerne prügeln,
kann man das nicht über Küchengeräte behaupten. Um solche Zustände und
Eigenschaften von Objekten repräsentieren zu können gibt es Prädikate.
Prädikate können entweder wahr oder falsch sein. Außerdem
unterscheidet man Prädikate nach ihrer Stelligkeit. Wir wollen uns
hier nur mit ein- und zwei-stelligen Prädikaten befassen. Ein Prädikat
ist einstellig, wenn es nur ein Parameter gibt. Die Eigenschaft
schizophren zu sein, kann durch ein einstelliges Prädikat modelliert
werden. Das Prädikat könnte heißen istSchizophren? und es hat nur
einen Parameter, nämlich ein Objekt vom Typen Human. In PDDL-Notation
sähe das dann so aus:

\begin{verbatim}
(istSchizophren? ?x – human)
\end{verbatim}

Wobei ?x den Parameter ersetzt und festgelegt werden muss von welchem
Typ dieser sein soll. Der momentane Ort einer Person hingegen, kann
durch ein zweistelliges Prädikat modelliert werden. Ein solches
Prädikat könnte heißen ort? und wäre durch zwei Parameter, nämlich ein
Objekt vom Typ Human und ein Objekt vom Typ Location, definiert. Das
würde in PDDL-Notation so aussehen:

\begin{verbatim}
(ort? ?x – human ?y – location)
\end{verbatim}

Analog wollen wir für unsere Beispielwelt folgende Prädikate
definieren:

\begin{enumerate}
\item Menschen können in Gebäuden und Räumen sein (Wo ist ein eine Person?)
\item Fahrzeuge müssen im selben Gebäude sein wie der Mensch der sie nutzen will (Ist ein Fahrzeug im selben Gebäude?)
\item Fahrzeuge müssen dem Menschen gehören der sie nutzen will (Gehört das Fahrzeug dem Mensch?)
\item Fahrzeuge können kaputt sein (Ist ein Fahrzeug gerade kaputt?)
\end{enumerate}

Versuche doch zunächst einmal selbst diese Prädikate in PDDL-Notation
aufzustellen.

\begin{verbatim}
(:predicates (at? ?person – human ?place – location)
(inSameBuilding? ?person – human ?veh – vehicle)
(belongsTo? ?veh – vehicle ?person – human)
(broken? ?veh – vehicle))
\end{verbatim}

Hier sieht man nun, dass es sinnvoll ist, Typen zu deklarieren, da
manche Prädikate auf Elemente von Oberklassen zutreffen können und
somit nicht alle Untertypen einzeln aufgeführt werden müssen.

Bisher sieht unsere Domäne so aus:

\begin{verbatim}
(define (domain fightclub)
	(:requirements :typing)
	(:types man woman – human
	human robot – agent 
	car bike – vehicle
	office university home – building
	kitchen livingroom bedroom garage bathroom – room
	room – home 
	building  – location 
	agent vehicle location – object)
	(:predicates (at? ?person – human ?place – location)
(inSameBuilding? ?person – human ?veh – vehicle)
(belongsTo? ?veh – vehicle ?person – human)
(broken? ?veh – vehicle))
\end{verbatim}

\subsubsection{Aktionen}
\label{sec-2-3-5}

\begin{verbatim}
(:action actionName2
:parameters (...)
:precondition (...)
:effect (...) )
... )
\end{verbatim}

Zu guter Letzt wollen wir uns nun den Aktionen widmen. Aktionen sind
wichtig, da durch Aktionen die Zustände und Eigenschaften von Objekten
derart verändert werden können, dass Probleme gelöst werden können.
Aktionen bestehen immer aus drei Teilen: Parametern, Vorbedingungen
und Wirkungen. Nehmen wir an, man könnte schlau werden durch lesen
bestimmter Inhalte. Damit hätten wir eine sehr einfache Aktion lesen.
Diese hätte zwei Parameter, nämlich die Person, die liest und die
Inhalte, die sie liest. Als Vorbedingung muss gelten, dass die Person
lesen kann, dass die Person nicht bereits schlau ist und dass die
Inhalte bilden (alles drei Prädikate, die bereits im Absatz predicates
auftauchen sollten). Der Effekt soll schließlich sein, dass die Person
schlau ist.

\begin{verbatim}
(:action read
         	:parameters 	(?p - person ?con - contents)
         	:precondition	(and (literate? ?p)
                             		(not (smart? ?p))
                            			(educating? ?con))
         	:effect 		(smart? ?p))
\end{verbatim}

Versuche nun einmal die Aktion drive für unsere Beispielwelt
aufzustellen, bei der eine Person ihren Ort ändert mittels eines
Fahrzeugs, das nicht kaputt sein darf, der Person gehören muss und im
selben Gebäude wie die Person zu Beginn sein muss.

\begin{verbatim}
(:action drive
        :parameters (?p - person ?v – vehicle ?to ?from – location)
        :precondition (and (at? ?p ?from)
                           (inSameBuilding? ?p ?v)
                           (belongsTo? ?v ?p)                                      
                           (not (broken? ?v)))
        :effect                 (at? ?p ?to))
\end{verbatim}

Herzlichen Glückwunsch! Wir haben gemeinsam eine Domäne erstellt.
Diese sieht jetzt so aus:

\begin{verbatim}
(define (domain fightclub)

  (:requirements :typing)

  (:types man woman – human
        human robot – agent 
        car bike – vehicle
        office university home – building
        kitchen livingroom bedroom garage bathroom – room
        room – home 
        building  – location 
        agent vehicle location – object)

  (:predicates (at? ?person – human ?place – location)
                (inSameBuilding? ?person – human ?veh – vehicle)
                (belongsTo? ?veh – vehicle ?person – human)
                (broken? ?veh – vehicle))

 (:action drive
     :parameters (?p - person ?v – vehicle ?to ?from – location)
     :precondition (and (at? ?p ?from)
                        (inSameBuilding? ?p ?v)
                        (belongsTo? ?v ?p)                                      
                        (not (broken? ?v)))
     :effect            (at? ?p ?to)))
\end{verbatim}

Hiermit hast du das Videotutorial geschafft. Keine Sorge, schwieriger
wird es nicht mehr! Solltest du noch Fragen haben, oder falls
irgendetwas noch unklar ist bzw. nicht ausreichend erklärt wurde,
wende dich bitte an den Versuchsleiter. Für den Rest des Versuchs ist
es wirklich wichtig, dass keine Fragen mehr offen sind!
% Emacs 24.3.1 (Org mode 8.2.5h)
\end{document}